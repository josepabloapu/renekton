\documentclass{beamer}
\usepackage[british]{babel}
\usepackage[utf8]{inputenc}
\usepackage{graphicx,hyperref,ucr_eie,url}
\usepackage{mdframed}
\usepackage{tikz}
\usetikzlibrary{snakes}
\usepackage{chronology}
\usepackage{listings}
\graphicspath{{../multimedia/images/}}
\definecolor{shadecolor}{RGB}{250,250,250}		%%boxes color
\lstset{
		basicstyle=\ttfamily,
		keywordstyle=\color{black},
		commentstyle=\color{black},
		stringstyle=\color{black},
		tabsize=2,
		backgroundcolor=\color{shadecolor}}

% The title of the presentation:
%  - first a short version which is visible at the bottom of each slide;
%  - second the full title shown on the title slide;
\title[Use of OpenCV to Send Commands]{7} 

% Optional: a subtitle to be dispalyed on the title slide
%\subtitle{}

% The author(s) of the presentation:
%  - again first a short version to be displayed at the bottom;
%  - next the full list of authors, which may include contact information;
\author[Jose Pablo Apú - B10407]{
  Francisco Mata\\
  Jose Pablo Apú Picado\\
  Sebastian Ramírez Solano\\\medskip
  %{\small \url{p.vullers@cs.ru.nl}} \\ 
  %{\small \url{http://www.cs.ru.nl/~pim/}}}
  }

% The institute:
%  - to start the name of the university as displayed on the top of each slide
%    this can be adjusted such that you can also create a Dutch version
%  - next the institute information as displayed on the title slide
\institute[University of Costa Rica]{
  Electrical Engineering School \\
  IE-0117 - Programación Bajo Plataformas Abiertas}

% Add a date and possibly the name of the event to the slides
%  - again first a short version to be shown at the bottom of each slide
%  - second the full date and event name for the title slide
\date[\today]{
  2nd Development Project \\
  \today}

\begin{document}

\begin{frame}
  \titlepage
\end{frame}

\begin{frame}
  \frametitle{Outline}
  \tableofcontents
\end{frame}

% Section titles are shown in at the top of the slides with the current section 
% highlighted. Note that the number of sections determines the size of the top 
% bar, and hence the university name and logo. If you do not add any sections 
% they will not be visible.
%%*****************************************************************************
\section{Introduction}

\begin{frame}
\frametitle{Objectives}

\end{frame}
%%*****************************************************************************
\begin{frame}
\frametitle{Justification}

\end{frame}
%%*****************************************************************************
\section{Requirements}

\begin{frame}
\frametitle{Hardware}
\includegraphics[scale=0.4]{requirements_hardware.jpg}
\end{frame}
%%*****************************************************************************
\begin{frame}
\frametitle{Software}
\begin{center}
\includegraphics[scale=0.3]{requirements_libs.jpg}
\end{center}
\end{frame}
%%*****************************************************************************
\section{Computer Vision}

\begin{frame}
\frametitle{Main concepts}
\end{frame}
%%*****************************************************************************
\begin{frame}
\frametitle{Typical tasks of computer vision}

\end{frame}
%%*****************************************************************************
\section{Implementation}
\begin{frame}
\frametitle{}
\end{frame}
%%*****************************************************************************
\begin{frame}
\frametitle{}
\end{frame}
%%*****************************************************************************
\begin{frame}
\frametitle{}
\end{frame}
%%*****************************************************************************
\begin{frame}
\frametitle{}
\end{frame}
%%*****************************************************************************
\begin{frame}
\frametitle{}
\end{frame}
%%*****************************************************************************
\section{Prototype}
\begin{frame}[fragile]
\frametitle{Pong}
\begin{lstlisting}
c code here
\end{lstlisting}
\end{frame}
%%*****************************************************************************
\begin{frame}[fragile]
\frametitle{Programming Interface - Example}
\begin{lstlisting}
c code here
\end{lstlisting}
\end{frame}
%%*****************************************************************************
\section{Conclusion \& Questions}
\begin{frame}
 \frametitle{Conclusion}
\begin{itemize}
\item Excelent for Text Processing
\item It is Pearl better than the other ones?
%\item Documentation
\end{itemize}
%Moose
%fork
\end{frame}
%%*****************************************************************************
\begin{frame}
 \frametitle{Questions?}
\end{frame}
%%*****************************************************************************
\begin{frame}
 \frametitle{References}
\begin{thebibliography}{10}
\bibitem{label}Marco, M. Marco, J (2010) Escanenado la Infomática. Editorial UOC. 
\end{thebibliography}
\end{frame}
%%*****************************************************************************

\end{document}
